
%%%%%%%%%%%%%%%%%%%%%%%%%%%%%%%%%%%%%%%%%
% Journal Article
% LaTeX Template
% Version 1.0 (25/8/12)
%
% This template has been downloaded from:
% http://www.LaTeXTemplates.com
%
% Original author:
% Frits Wenneker (http://www.howtotex.com)
%
% License:
% CC BY-NC-SA 3.0 (http://creativecommons.org/licenses/by-nc-sa/3.0/)
%
%%%%%%%%%%%%%%%%%%%%%%%%%%%%%%%%%%%%%%%%%

%----------------------------------------------------------------------------------------
%	PACKAGES AND OTHER DOCUMENT CONFIGURATIONS
%----------------------------------------------------------------------------------------

%\documentclass[paper=a4, fontsize=11pt]{scrartcl} % A4 paper and 11pt font size
\documentclass[paper=a4, fontsize=11pt]{article} % A4 paper and 11pt font size
%\documentclass[twoside]{article}

\usepackage{graphicx}
\usepackage{ucs}
\usepackage[utf8x]{inputenc}
\usepackage[frenchb]{babel}
%\usepackage[sc]{mathpazo} % Use the Palatino font
\usepackage[T1]{fontenc} % Use 8-bit encoding that has 256 glyphs
\usepackage{microtype} % Slightly tweak font spacing for aesthetics
\usepackage[hmarginratio=1:1,top=32mm,columnsep=20pt]{geometry} % Document margins
\usepackage[colorlinks=true, urlcolor=cyan, pdftitle={Alsace Reseau Neutre}]{hyperref}
\usepackage[hang, small,labelfont=bf,up,textfont=it,up]{caption} % Custom captions under/above floats in tables or figures
\usepackage{booktabs} % Horizontal rules in tables
%\usepackage{float} % Required for tables and figures in the multi-column environment - they need to be placed in specific locations with the [H] (e.g. \begin{table}[H])
%\usepackage{lettrine} % The lettrine is the first enlarged letter at the beginning of the text
\usepackage{paralist} % Used for the compactitem environment which makes bullet points with less space between them
\usepackage{fancyhdr} % Headers and footers
\usepackage{titlesec} % Allows customization of titles
\usepackage{abstract} % Allows abstract customization

\newcommand{\horrule}[1]{\rule{\linewidth}{#1}} % Create horizontal rule command with 1 argument of height
\renewcommand{\abstractnamefont}{\normalfont\bfseries} % Set the "Abstract" text to bold
\renewcommand{\abstracttextfont}{\normalfont\small\itshape} % Set the abstract itself to small italic text

\linespread{1.1} % Line spacing - Palatino needs more space between lines
%\titleformat{\section}[block]{\large\scshape\centering{\Roman{section}.}}{}{1em}{} % Change the look of the section titles 

\pagestyle{fancy} % All pages have headers and footers
\fancyhead{} % Blank out the default header
\fancyfoot{} % Blank out the default footer
\fancyhead[C]{Fédération French Data Network $\bullet$ Alsace Réseau Neutre } % Custom header text
\fancyfoot[RO,LE]{\thepage} % Custom footer text

%
%\allsectionsfont{\centering \normalfont\scshape} % Make all sections centered, the default font and small caps
%\setlength\parindent{0pt} % Removes all indentation from paragraphs - comment this line for an assignment with lots of text
%

%----------------------------------------------------------------------------------------
%	TITLE SECTION
%----------------------------------------------------------------------------------------

\title{\vspace{-15mm}\fontsize{20pt}{2pt}\selectfont\textbf{Alsace Réseau Neutre}} % Article title
\date{} % Date pour virer la date

%----------------------------------------------------------------------------------------

\begin{document}

\maketitle % Insert title
\thispagestyle{fancy} % La première page a aussi un footer et un header

\section{Introduction et explications}
Notre association est l'un des fournisseurs d'accès à Internet associatifs qui 
respectent les principes de neutralité du net en France et 
qui fait partie de la Fédération FDN.

\subsection{Qu'est-ce que la Fédération FDN}
La Fédération FDN (FFDN) regroupe les Fournisseurs d'Accès à Internet 
associatifs respectant la neutralité du net,
French Data Network étant le FAI le plus ancien (plus de 20 ans) encore en activité.

\subsection{Ce qu'est un Fournisseur d'Accès à Internet}
Le Fournisseur d'Accès à Internet est un simple tuyau qui vous relie à un 
centre de données (un « Data Center »). 
De là, il transmet les données sur d'autres réseaux (d'autres FAI par exemple) 
jusqu'à arriver au bon destinataire.

\subsection{Ce qu'est la neutralité du net}
Le réseau d'un FAI est « neutre » s'il transmet les données suivant ces 4 critères :

\begin{itemize}
\item sans en examiner le contenu ;
\protect\footnote{
	Le FAI n'a pas besoin de savoir que vous regardez une vidéo ou que vous jouez 
		à un jeu vidéo pour faire son travail, alors il n'a pas à le savoir.
	Cette information relève de la vie privée.
}
\item sans prise en compte de la source ou de la destination des données ;
\protect\footnote{
	Exemple : on ne privilégie pas youtube à dailymotion.
}
\item sans privilégier un protocole de communication ;
\protect\footnote{
	Exemple : le web qui serait plus rapide que la messagerie instantanée ou l'e-mail.
}
\item sans en altérer le contenu.
\protect\footnote{
	Aucune censure ni « pourrissement » quelconque ne sera fait sur votre connexion.
}
\end{itemize}

\subsection{Ce qu'apporte le modèle associatif}
\subsubsection{Une assurance}
Il n'y a pas de volonté économique.
Personne ne gagne de l'argent ce qui assure d'une certaine manière que nous 
n'avons aucun intérêt à recueillir des informations personnelles pour les revendre, 
ou limiter votre connexion de quelque manière que ce soit.

\subsubsection{Le contrôle}
Vous êtes votre fournisseur d'accès à Internet.
À partir du moment où vous êtes adhérent, vous avez une voix qui compte tout 
autant qu'une autre dans notre organisation.
Toute organisation souhaitant s'inscrire parmi les membres de FFDN doit avoir 
un système démocratique.

\subsubsection{L'apprentissage}
Toute personne peut venir apprendre comment est fait un fournisseur d'accès à Internet, 
	comment le faire et le maintenir en activité, 
	les technologies employées, etc. 
	% Présentation de la Fédération, explications diverses (FAI, Neutralité)
%\clearpage
\section{Alsace Réseau Neutre dans le détail}

\subsection{Pourquoi ARN}
Il n'existait à ce jour aucun FAI associatif en Alsace et le côté 
formateur a attiré suffisamment de personnes pour que le projet soit lancé.

\subsection{Les services proposés par ARN}
En plus de l'accès à Internet, nous proposons déjà un certain nombre de 
services utiles pour nos adhérents.
\protect\footnote{
	Pour une liste détaillée, veuillez consulter notre site web : 
		\href{http://www.arn-fai.net}{www.arn-fai.net} .
}
Voici une petite liste non exhaustive :
\begin{itemize}
	\item Un PAD, pour de l'édition collaborative de documents ;
	\item Le nom de domaine « arn-fai.net » avec lequel nous pouvons faire 
	des enregistrements de type A ou AAAA.
	\protect\footnote{
		Exemple : jean.arn-fai.net pointe sur l'IP de l'adhérent jean.
	}
	\item L'hébergement web pour qui le souhaite ;
	\protect\footnote{
		Venez simplement demander sur IRC ou par e-mail pour avoir votre site web en ligne. 
		Aucune garantie sur la disponibilité, l'hébergement se fait chez d'autres adhérents. 
		Nous privilégions l'auto-hébergement, on vous y aide au besoin.
	}
	\item Un service de gestion de projets (versionnement) ;
	\protect\footnote{
		Outil de versionnement : GIT. 
		Gestionnaire de dépôts : gitosis (actuellement). GitLab d'ici peu.
	}
	\item Un service de "paste", qui permet de mettre en ligne un petit bout 
	de configuration ou de code sans polluer un canal IRC ou un forum.
\end{itemize}

\subsection{Le futur d'ARN}
Tout projet s'inscrivant dans la logique de préservation de la neutralité du net 
et du renforcement du réseau Internet (et de son accès) peut être envisagé.
\subsubsection{Les projets envisagés}
Nous discutons actuellement de la mise en place d'un réseau maillé de points 
d'accès wifi sur Strasbourg et ses environs afin de proposer un accès à internet 
à bas prix, et d'améliorer la redondance de ce service.
\protect\footnote{
	Si un accès est coupé (travaux par exemple), l'adhérent garde toujours son 
		accès à Internet.
}

Dans un tout autre registre, nous pouvons utiliser nos serveur DNS
\protect\footnote{
	Domain Name Service : serveur de nom de domaine.
	Permet de nommer les machines et de ne plus avoir à retenir leur adresse 
	IP, exemple : www.arn-fai.net au lieu de 85.160.35.17 (relativement 
			barbare à retenir, et change régulièrement).
}
	pour créer un TLD
\protect\footnote{
	Top Level Domain, exemple « .org » ou « .fr ».
}
	 « .arn » accessible uniquement depuis nos DNS.
\protect\footnote{
	Exemple : création du nom de domaine « jean-dupont.arn » ou « the-killer-du-67.arn ».
}

\subsubsection{De futurs services}
Nous pourrons à l'avenir mettre en place des services payants tels que :
\begin{itemize}
	\item Un hébergement plus général de services (VPS éventuellement) ;
	\item Un Virtual Private Network ;
\end{itemize}
Ceux-ci permettront de financer des projets nécessitant l'achat de matériel.

\subsection{Nous contacter - nouvelles du projet}
Vous pouvez nous contacter par e-mail via notre liste de diffusion 
\href{mailto:discussion@listes.arn-fai.net}{discussion@listes.arn-fai.net} .
Nous nous en servons également pour partager les nouvelles de l'association 
dans les grandes lignes.

Si vous souhaitez discuter en direct avec nous, vous pouvez venir sur notre 
salon IRC : \#arn sur irc.geeknode.org .
Pour contacter directement le président : 
\href{mailto:president@arn-fai.net}{president@arn-fai.net} .
Le trésorier : \href{mailto:tresorier@arn-fai.net}{tresorier@arn-fai.net}, 
   le secrétaire : \href{mailto:secretaire@arn-fai.net}{secretaire@arn-fai.net} .
N'hésitez pas à consulter notre site web : \href{http://www.arn-fai.net}{www.arn-fai.net}.

		% Spécifique à NOTRE FAI

\end{document}
