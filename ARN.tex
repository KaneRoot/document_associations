\section{Alsace Réseau Neutre dans le détail}

\subsection{Pourquoi ARN}
Il n'existait à ce jour aucun FAI associatif en Alsace et le côté 
formateur a attiré suffisamment de personnes pour que le projet soit lancé.

\subsection{Les services proposés par ARN}
En plus de l'accès à Internet, nous proposons déjà un certain nombre de 
services utiles pour nos adhérents.
\protect\footnote{
	Pour une liste détaillée, veuillez consulter notre site web : 
		\href{http://www.arn-fai.net}{www.arn-fai.net} .
}
Voici une petite liste non exhaustive :
\begin{itemize}
	\item Un PAD, pour de l'édition collaborative de documents ;
	\item Le nom de domaine « arn-fai.net » avec lequel nous pouvons faire 
	des enregistrements de type A ou AAAA.
	\protect\footnote{
		Exemple : jean.arn-fai.net pointe sur l'IP de l'adhérent jean.
	}
	\item L'hébergement web pour qui le souhaite ;
	\protect\footnote{
		Venez simplement demander sur IRC ou par e-mail pour avoir votre site web en ligne. 
		Aucune garantie sur la disponibilité, l'hébergement se fait chez d'autres adhérents. 
		Nous privilégions l'auto-hébergement, on vous y aide au besoin.
	}
	\item Un service de gestion de projets (versionnement) ;
	\protect\footnote{
		Outil de versionnement : GIT. 
		Gestionnaire de dépôts : gitosis (actuellement). GitLab d'ici peu.
	}
	\item Un service de "paste", qui permet de mettre en ligne un petit bout 
	de configuration ou de code sans polluer un canal IRC ou un forum.
\end{itemize}

\subsection{Le futur d'ARN}
Tout projet s'inscrivant dans la logique de préservation de la neutralité du net 
et du renforcement du réseau Internet (et de son accès) peut être envisagé.
\subsubsection{Les projets envisagés}
Nous discutons actuellement de la mise en place d'un réseau maillé de points 
d'accès wifi sur Strasbourg et ses environs afin de proposer un accès à internet 
à bas prix, et d'améliorer la redondance de ce service.
\protect\footnote{
	Si un accès est coupé (travaux par exemple), l'adhérent garde toujours son 
		accès à Internet.
}

Dans un tout autre registre, nous pouvons utiliser nos serveur DNS
\protect\footnote{
	Domain Name Service : serveur de nom de domaine.
	Permet de nommer les machines et de ne plus avoir à retenir leur adresse 
	IP, exemple : www.arn-fai.net au lieu de 85.160.35.17 (relativement 
			barbare à retenir, et change régulièrement).
}
	pour créer un TLD
\protect\footnote{
	Top Level Domain, exemple « .org » ou « .fr ».
}
	 « .arn » accessible uniquement depuis nos DNS.
\protect\footnote{
	Exemple : création du nom de domaine « jean-dupont.arn » ou « the-killer-du-67.arn ».
}

\subsubsection{De futurs services}
Nous pourrons à l'avenir mettre en place des services payants tels que :
\begin{itemize}
	\item Un hébergement plus général de services (VPS éventuellement) ;
	\item Un Virtual Private Network ;
\end{itemize}
Ceux-ci permettront de financer des projets nécessitant l'achat de matériel.

\subsection{Nous contacter - nouvelles du projet}
Vous pouvez nous contacter par e-mail via notre liste de diffusion 
\href{mailto:discussion@listes.arn-fai.net}{discussion@listes.arn-fai.net} .
Nous nous en servons également pour partager les nouvelles de l'association 
dans les grandes lignes.

Si vous souhaitez discuter en direct avec nous, vous pouvez venir sur notre 
salon IRC : \#arn sur irc.geeknode.org .
Pour contacter directement le président : 
\href{mailto:president@arn-fai.net}{president@arn-fai.net} .
Le trésorier : \href{mailto:tresorier@arn-fai.net}{tresorier@arn-fai.net}, 
   le secrétaire : \href{mailto:secretaire@arn-fai.net}{secretaire@arn-fai.net} .
N'hésitez pas à consulter notre site web : \href{http://www.arn-fai.net}{www.arn-fai.net}.

